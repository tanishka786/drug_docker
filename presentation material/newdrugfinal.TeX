\documentclass[conference]{IEEEtran}
\IEEEoverridecommandlockouts

\usepackage{cite}
\usepackage{amsmath,amssymb,amsfonts}
\usepackage{algorithmic}
\usepackage{graphicx}
\usepackage{textcomp}
\usepackage{xcolor}
\usepackage{hyperref}
\def\BibTeX{{\rm B\kern-.05em{\sc i\kern-.025em b}\kern-.08em
    T\kern-.1667em\lower.7ex\hbox{E}\kern-.125emX}}
\begin{document}

\title{Blockchain for Pharmaceutical Supply Chain Integrity: A Decentralized Approach to Combat Counterfeit Drugs\\}

\author{
\IEEEauthorblockN{1\textsuperscript{st} Brinal Colaco}
\IEEEauthorblockA{\textit{Department of Computer Engineering}\\
\textit{Vidyavardhini's College of Engineering and Technology}\\
Mumbai, Maharashtra \\
brinalcolaco@gmail.com}
\hfill
\vspace{2\baselineskip} 
\IEEEauthorblockN{2\textsuperscript{nd} Tanishka Das}
\IEEEauthorblockA{\textit{Department of Computer Engineering}\\
\textit{Vidyavardhini's College of Engineering and Technology}\\
Mumbai, Maharashtra \\
tanishkajdas.ce@gmail.com}
\and
\IEEEauthorblockN{3\textsuperscript{rd} Pallavi Dhandar}
\IEEEauthorblockA{\textit{Department of Computer Engineering}\\
\textit{Vidyavardhini's College of Engineering and Technology}\\
Mumbai, Maharashtra \\
rajendrasdhandar@gmail.com}
\hfill
\vspace{2\baselineskip} 
\IEEEauthorblockN{4\textsuperscript{th} Jatush Hingu}
\IEEEauthorblockA{\textit{Department of Computer Engineering}\\
\textit{Vidyavardhini's College of Engineering and Technology}\\
Mumbai, Maharashtra \\
jatushhingu29@gmail.com}
\vspace{2\baselineskip} 
\and
\begin{minipage}{\textwidth}
\centering
% \IEEEauthorblockA{5\textsuperscript{th} Akash Nadar}
\IEEEauthorblockA{\textit{Department of Computer Engineering}\\
\IEEEauthorblockA{5\textsuperscript{th} Akash Nadar}
\textit{Vidyavardhini's College of Engineering and Technology}\\
Mumbai, Maharashtra \\
akashnadar102@gmail.com}
\end{minipage}
}
% \author{
% \IEEEauthorblockN{Brinal Colaco}  
% \IEEEauthorblockA{\textit{Department of Computer Engineering} \\  
% \textit{Vidyavardhini's College of Engineering and Technology} \\  
% Mumbai, Maharashtra \\  
% 0000-0003-2937-9790}  
% \and  
% \IEEEauthorblockN{Tanishka Das}  
% \IEEEauthorblockA{\textit{Department of Computer Engineering} \\  
% \textit{Vidyavardhini's College of Engineering and Technology} \\  
% Mumbai, Maharashtra \\  
% 0009-0006-0632-8877}  
% \and  
% \IEEEauthorblockN{Pallavi Dhandar}  
% \IEEEauthorblockA{\textit{Department of Computer Engineering} \\  
% \textit{Vidyavardhini's College of Engineering and Technology} \\  
% Mumbai, Maharashtra \\  
% 0009-0000-6301-7174}  
% \and  
% \IEEEauthorblockN{Jatush Hingu}  
% \IEEEauthorblockA{\textit{Department of Computer Engineering} \\  
% \textit{Vidyavardhini's College of Engineering and Technology} \\  
% Mumbai, Maharashtra \\  
% email address or ORCID}  
% \and  
% \IEEEauthorblockN{Akash Nadar}  
% \IEEEauthorblockA{\textit{Department of Computer Engineering} \\  
% \textit{Vidyavardhini's College of Engineering and Technology} \\  
% Mumbai, Maharashtra \\  
% email address or ORCID}  
% }

%
\maketitle
%
\begin{abstract}
This study introduces a blockchain-based solution for tracking the pharmaceutical drug supply chain, enhancing transparency, security, and authenticity from production to final sale. By leveraging Ganache for blockchain simulation and MetaMask for account management, we establish a decentralized ledger that immutably records every transaction in a drug's lifecycle.

The system monitors each stage, from raw material sourcing to manufacturing, distribution, and retail, enabling authorized participants to update and track drug movement in real time. Smart contracts automate the verification process, ensuring compliance with predefined conditions at every stage.

This approach not only safeguards supply chain integrity but also provides a reliable audit trail for regulatory authorities, mitigating the risk of counterfeit drugs. End users, including consumers and pharmacies, can verify drug authenticity and safety, fostering a more secure and trustworthy pharmaceutical ecosystem.
\end{abstract}

\begin{IEEEkeywords}
Blockchain, Pharmaceutical Supply Chain, Transparency, Security, Smart Contracts, Counterfeit Prevention.
\end{IEEEkeywords}

\section{Introduction}
Keeping drugs safe, legitimate, and intact along the supply chain is a major concern for the pharmaceutical sector. Problems including mishandled drugs, counterfeit drugs, and a lack of transparency present major health hazards to patients.  Traditional supply chain systems often fail to provide the necessary transparency and traceability, making it difficult to ensure drug safety from manufacturing to end-user distribution.This project presents a blockchain-powered solution that establishes a transparent, decentralized, and immutable ledger to securely track every transaction in the pharmaceutical supply chain. By integrating Ganache for blockchain simulation and MetaMask for account management, the system enables seamless stakeholder interactions while ensuring real-time tracking of drug movement—from raw materials to manufacturing, distribution, and final retail.

At the heart of this system are smart contracts, which automate and enforce verification at every point of the supply chain, reducing human error and assuring secure, timely transfers. The technology improves not only openness and traceability but also regulatory compliance by providing an immutable audit trail for health authorities. This considerably reduces the possibility of fake medications getting into the supply chain.This blockchain-based method ensures accountability at every stage by giving all parties involved—manufacturers, distributors, pharmacies, and regulators—access to real-time, tamper-proof data. The ability to quickly confirm the security and legitimacy of the medications they buy benefits end users and promotes confidence in the pharmaceutical industry. The fragmentation, inefficiencies, and lack of transparency in the current pharmaceutical supply chain techniques increase the risk of counterfeit goods and create regulatory obstacles. Outdated paper-based documentation and siloed digital systems hinder real-time tracking and data visibility, making it challenging to verify drug authenticity and comply with strict regulations such as DSCSA and FMD.

A blockchain-based solution revolutionizes the pharmaceutical supply chain by introducing a secure, decentralized ledger that enables:

\begin{itemize}
    \item Real-time tracking of drugs at every stage.
    \item Enhanced drug authenticity verification to prevent counterfeits.
    \item Streamlined compliance with regulatory standards.
    \item Improved operational efficiency across the supply chain.
\end{itemize}

By securing the entire lifecycle of pharmaceuticals, this system drastically reduces the risk of counterfeit drugs, enhances regulatory adherence, and builds a safer, more reliable, and efficient pharmaceutical supply chain. 

To further strengthen security, the project also implements a college student ID-based verification system, ensuring that only authorized students from specific campuses can access the application. This feature prevents unauthorized access to sensitive data, reinforcing security while maintaining a seamless user experience.
This blockchain-driven approach not only transforms supply chain management but also sets a new standard for security, transparency, and trust in the pharmaceutical industry.


\section{Literature Survey}
This study introduces an advanced blockchain-based approach for improving transparency, security, and traceability in the pharmaceutical supply chain. The suggested solution guarantees safe medication transportation from the acquisition of raw materials to retail distribution by utilizing Ethereum smart contracts, Ganache for blockchain simulation, and MetaMask for account administration. In contrast to traditional frameworks, this decentralized method reduces vulnerabilities, creates an unchangeable audit trail, guarantees regulatory compliance, and increases consumer trust in pharmaceutical products. 
[1].
The use of blockchain technology in supply chain management has been the subject of numerous investigations. Research on using Hyperledger Fabric for secure transactions has shown promise in enhancing pharmaceutical logistics security; nevertheless, these methods frequently lack scalability and real-time adaptation. [2, 3]. Moreover, technologies such as encrypted QR codes and RFID have been employed for drug traceability; however, they remain reliant on centralized verification mechanisms [4, 5]. Studies such as Drugledger have introduced service segmentation strategies to enhance data authenticity and privacy [6], while Medledger has utilized Hyperledger Fabric to establish permissioned blockchain networks for secure transactions [7]. Despite these advancements, many of these solutions remain theoretical, lack AI-driven automation, and are restricted to Windows-based environments, limiting their flexibility [8, 9].
The proposed system addresses these limitations by incorporating advanced technologies. Unlike prior studies, this research integrates Dockerization, enabling seamless deployment across Linux, macOS, and Windows, thereby ensuring cross-platform compatibility and enhanced accessibility. Additionally, the system leverages an AI-powered chatbot to provide real-time updates, automate responses, and facilitate decision-making for stakeholders . These innovations contribute to a more scalable, intelligent, and practical blockchain-based pharmaceutical tracking system, establishing a new standard for supply chain management.



\section{Methodology}
The blockchain-based pharmaceutical supply chain solution's implementation architecture consists of numerous components that work together to create a safe, efficient, and user-friendly platform for drug tracking. The architecture's modular form makes it easier to scale and maintain. The main elements and their interactions are depicted in the block figure further down, which also shows the application's methodology as shown in Fig. 1.

\begin{figure}[h]
    \centering
    \includegraphics[width=1\linewidth]{assets/enhanced_supplier.png}
    \caption{Workflow of the site.}
    \label{fig1}
\end{figure}

\subsection{Implement a Secure Tracking System} Develop a blockchain-based platform that securely records every transaction in the drug supply chain, ensuring that stakeholders can verify the authenticity and journey of pharmaceuticals from production to sale. This will enhance trust and accountability in the system [8].

\subsection{Blockchain Perform} The core application is where users (manufacturers, distributors, retailers, and regulators) interact with the pharmaceutical tracking system.

\subsection{ User Interface (UI)} Represents the graphical interface through which users navigate the application, view drug tracking information, initiate transactions, and access features like the knowledge base [8].

\subsection{User Authentication} Verifies user credentials, ensuring that only authorized stakeholders can access the system. This includes implementing secure login methods and access control.



\subsection{The system comprises the following actors:} 

\begin{itemize}
    \item Raw Material Supplier: Records the source and quality of raw materials.
    \item Manufacturer: Registers production details and quality checks.
    \item Distributor: Tracks logistics data and ensures temperature control.
    \item Retailer: Verifies product authenticity before selling.
\end{itemize}

The system leverages Ethereum Smart Contracts for secure and automated transactions. The Hyperledger Fabric framework is considered for permissioned networks where stakeholders have defined roles and access controls.

\subsection{Blockchain Database} The blockchain securely stores drug movements, user profiles, and compliance records, ensuring data integrity and transparency across the supply chain[9].



\section{Proposed system}
The flowchart demonstrates the Proposed system as shown in Fig. 2.

\begin{figure}
    \centering
    \includegraphics[width=1\linewidth]{assets/enhanced_final.png}
    \caption{Process Diagram of site.}
    \label{fig2}
\end{figure}



\subsection{User Enters the App}
The user accesses the app's splash screen

\subsection{Check for Existing Account}
\begin{itemize}
    \item  If the user already has an account, proceed with login process.
    \item  If not, display the register user screen.
\end{itemize}

\subsection{Login Process}

\begin{itemize}
    \item User enters their username/email and password.
    \item The system validates the credentials.
    \item If valid, redirect to the home screen.
    \item  If not, display an error message.
\end{itemize}

\subsection{Registration Process}
\begin{itemize}
    \item User clicks on the registration.
    \item User provides necessary information (e.g.,name, email, password).
    \item The system validates the information.
    \item If valid, create a new account and redirect to the home screen.
    \item If not, display an error message.
\end{itemize}

\subsection{Home Screen}
 After successfully logging in or registering, the user is redirected to the home screen.The home screen displays various roles, allowing users to select their specific functions such as
 Raw Produce,
 Manufacturer,
 Distributor,
 Retailer,
 Logout Button.

\subsection{Tracking Section}
In this section, the user can view a list of drugs along with their tracking information. Users can select a specific drug to view its journey through the supply chain, including current status and location.

\subsection{Logout }
User can log out from their account by clicking the logout button. After logging out, they are redirected to the login or registration page.



\section{Results}
To set up the system, first connect Metamask Fig. 4 to the Ganache blockchain by starting the workspace you created, and entering the Metamask password to initiate the connection. Ganache will then display 10 Ethereum accounts as shown in Fig. 3. In the settings, you can see the workspace details and confirm the connection with Truffle's configuration file, where the server connects to the specified host and port. Only the owner can perform registration by importing accounts from Ganache using private keys. The owner adds various actors like the raw material supplier, manufacturer, distributor, and retailer by registering their details, such as name, location, and Ethereum address.  


\begin{figure}[h]
    \centering
    \includegraphics[width=1\linewidth]{assets/Fig 5.5.png}
    \caption{Ganache Account details page.}
    \label{fig3}
\end{figure}

\begin{figure}[h]
    \centering
    \includegraphics[width=1\linewidth]{assets/Fig 5.10.png}
    \caption{Metamask Account of actors.}
    \label{fig4}
\end{figure}



After registering the actors, the owner can add the medicine to be manufactured. Each actor performs their specific task using their respective accounts, such as the supplier providing raw materials by adding the medicine ID. Once all tasks are completed, the medicine can be tracked by entering its ID, allowing any actor to check its status as needed. This setup ensures clear roles and responsibilities among all involved parties while enabling efficient tracking of medicines.



\section{Conclusion and Future scope}
The Drug Guardian platform improves the drug supply chain's efficiency, security, and transparency by leveraging blockchain technology. The technology guarantees that all parties involved—manufacturers, distributors, pharmacies, and patients—have access to precise and up-to-date information about the provenance and validity of drugs by utilizing decentralized ledgers. This strategy not only reduces the possibility of fake medications reaching the market, but it also increases user trust. The platform architecture addresses major difficulties in the pharmaceutical business, including traceability, regulatory compliance, and data integrity. Through continuous development, medication Guardian strives to improve the medication supply chain, contributing to better healthcare results and patient safety.

Improving Compatibility with Current Systems Future initiatives will prioritize integrating the Drug Guardian platform with existing healthcare systems, such as EHRs and pharmacy management systems. This will enhance stakeholder collaboration alongside information flow.
The Drug Guardian platform will evolve to ensure the integrity and safety of the drug supply chain, benefiting both patients and healthcare practitioners.


% GitHub repository link
For more details, visit the project repository at: \href{https://github.com/tanishka786/Drug_Blockchain.git}{GitHubRepository}.



\begin{thebibliography}{00}
\bibitem{ref_article1}
Saberi, S., Kouhizadeh, M., Sarkis, J., and Shen, L.: Blockchain technology and its relationships to sustainable supply chain management. International Journal of Production Research \textbf{57}(7), 2117--2135 (2018). 
\bibitem{ref_article2}
Roman-Belmonte, J. M., De la Corte-Rodriguez, H., and Rodriguez-Merchan, E. C.: 
How blockchain technology can change medicine. Postgraduate Medicine 
\textbf{130}(4), 420--427 (2018)
\bibitem{ref_article3}
Vruddhula, S.: Application of on-dose identification and blockchain to prevent drug counterfeiting. 
Pathogens and Global Health \textbf{112}(4), 161--161 (2018)
\bibitem{ref_article4}
Ghadge, A., Bourlakis, M., Kamble, S., and Seuring, S.: Blockchain implementation 
in pharmaceutical supply chains: A review and conceptual framework. 
International Journal of Production Research \textbf{61}(19), 6633--6651 (2023)
\bibitem{ref_article5}
Rai, B. K.: BBTCD: blockchain-based traceability of counterfeited drugs. 
Health Services and Outcomes Research Methodology \textbf{23}(3), 337--353 (2023)
\bibitem{ref_article6}
Kumar, M.: Blockchain Technology – A Algorithm for Drug Serialization. 
Universal Journal of Pharmacy and Pharmacology, pp. 61--67 (2022)
\bibitem{ref_article7}
Roman-Belmonte, J. M., De la Corte-Rodriguez, H., and Rodriguez-Merchan, E. C.: 
How blockchain technology can change medicine. Postgraduate Medicine 
\textbf{130}(4), 420--427 (2018)
\bibitem{ref_article8}
Haq, I., and Muselemu, O.: Blockchain Technology in Pharmaceutical Industry to Prevent Counterfeit Drugs. 
International Journal of Computer Applications \textbf{180}, 8--12 (2018), 
doi:10.5120/ijca2018916579
\bibitem{ref_article9}
Panda, S. K., and Satapathy, S. C.: Drug traceability and transparency in medical 
supply chain using blockchain for easing the process and creating trust between 
stakeholders and consumers. Personal and Ubiquitous Computing, pp. 1--17 (2021)
\bibitem{ref_article10}
Mokrova, L. P., Borodina, M. A., Goncharov, V. V., Popov, S. A., and Kepa, Y. N.: 
Prospects for using blockchain technology in healthcare: Supply chain management. 
Entomology and Applied Science Letters \textbf{8}(2), 71--77 (2021)

\bibitem{ref_proc1}
X. Xu, N. Tian, H. Gao, H. Lei, Z. Liu, and Z. Liu: A Survey on Application of Blockchain Technology 
in Drug Supply Chain Management. In: 2023 IEEE 8th International Conference on Big Data Analytics 
(ICBDA), pp. 62--71. Harbin, China (2023), doi:10.1109/ICBDA57405.2023.10104779
\bibitem{ref_proc2}
Z. Zheng, S. Xie, H. Dai, X. Chen, and H. Wang: An Overview of Blockchain Technology: 
Architecture, Consensus, and Future Trends. In: 2017 IEEE International Congress 
on Big Data (BigData Congress), pp. 557--564. Honolulu, HI, USA (2017).

\bibitem{ref_article11}
National Health Insurance Administration. Analysis of Drug Usage. NHIA, 2016. 

\bibitem{ref_article12}P. Toscan, The Dangerous World of Counterfeit Prescription Drugs, Jun. 2020, [online] Available: http://usatoday30.usatoday.com/money/industries/health/drugs/story/2011-10-09/cnbc-drugs/50690880/1.


\bibitem{ref_article13}A. Kawa and A. Maryniak, “Blockchain
Applications in Supply Chain”, SMART Supply
Network, pp. 21-46,2019.


\bibitem{ref_article14}Rodrigues U R 2019 Law and the blockchain Iowa Law Review 104 679-729

\bibitem{ref_article15}T. Ahram, A. Sargolzaei, S. Sargolzaei, J.
Daniels, and B. Amaba, “Blockchain technology
innovations”, In: Proc. of Technology 
Engineering Management Conf. (TEMSCON),
pp. 137-141, 2017.


\bibitem{ref_article16}M. Swan, Blockchain: Blueprint for a New Economy, Sebastopol, CA, USA:O’Reilly Media, 2015. 


\bibitem{ref_article17}G Venkat Sai Kumar, C Rohith Reddy, B.M Beena, "Revolutionizing Healthcare Waste Management: Leveraging Blockchain For Enhanced Efficiency and Transparency", 2024 15th International Conference on Computing Communication and Networking Technologies (ICCCNT), pp.1-5, 2024.


\bibitem{ref_article18}Paula Fraga-Lamas, Tiago M. Fernández-Caramés, António M. Rosado da Cruz, Sergio Ivan Lopes, "An Overview of Blockchain for Industry 5.0: Towards Human-Centric, Sustainable and Resilient Applications", IEEE Access, vol.12, pp.116162-116201, 2024.


\bibitem{ref_article19}Lakshya Gupta, Sahil Girotra, Jashmeet Singh Lehal, Krishna Sharma, Suhaib Ahmed, "Investigation into Applicability of Blockchain Technology in Healthcare Systems", 2023 7th International Conference On Computing, Communication, Control And Automation (ICCUBEA), pp.1-6, 2023.

\bibitem{ref_article20}Christidis, K. and Devetsikiotis, M. (2016), “Blockchains and smart contracts for the Internet of Things”, IEEE Access, Vol. 4, pp. 2292-2303.

\bibitem{ref_article21}S. Tönnissen and F. Teuteberg, "Analysing the impact of blockchain-technology for operations and supply chain management: An explanatory model drawn from multiple case studies", Int. J. Inf. Manage., vol. 52, Jun. 2020.


\bibitem{ref_article22}S. Kummer, D. M. Herold, M. Dobrovnik, J. Mikl and N. Schäfer, "A systematic review of blockchain literature in logistics and supply chain management: Identifying research questions and future directions", Future Internet, vol. 12, no. 3, pp. 60, Mar. 2020.


\bibitem{ref_article23}Durach, C.F., Blesik, T., von Düring, M. and Bick, M. (2021), “Blockchain applications in supply chain transactions”, Journal of Business Logistics, Vol. 42 No. 1, pp. 7-24.

\bibitem{ref_article24}N. Nizamuddin, K. Salah, M. Ajmal Azad, J. Arshad and M. H. Rehman, "Decentralized document version control using ethereum blockchain and IPFS", Comput. Electr. Eng., vol. 76, pp. 183-197, Jun. 2019.


\bibitem{ref_article25}Hasselgren, A., Kralevska, K., Gligoroski, D., Pedersen, S.A. and Faxvaag, A. (2020), “Blockchain in healthcare and health sciences—a scoping review”, International Journal of Medical Informatics, Vol. 134, 104040.


\bibitem{ref_article26}B. Alangot and K. Achuthan, “Trace and Track:
Enhanced Pharma Supply Chain Infrastructure
to Prevent Fraud”, In: Proc. of International
Conf. on Ubiquitous Communications and
Network Computing, pp. 189-195, 2017.



\bibitem{ref_article27}Casado-Vara R, González-Briones A, Prieto J and Corchado J M 2019 Smart Contract for Monitoring and Control of Logistics Activities: Pharmaceutical Utilities Case Study Advances in Intelligent Systems and Computing 771 509-517

\bibitem{ref_article28}J. Moosavi, "Blockchain in supply chain management: A review bibliometric and network analysis", Environ. Sci. Pollut. Res., vol. 2021, pp. 1-15, Feb. 2021.


\bibitem{ref_article29}S. Oudaya Coumar, R. Surender, S. Yazhinian, "Efficient Supply Chain Management System Using ARM CORTEX-M3", 2023 International Conference on System, Computation, Automation and Networking (ICSCAN), pp.1-4, 2023.


\end{thebibliography}
\end{document}